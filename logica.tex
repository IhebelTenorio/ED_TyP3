\documentclass[11pt,letterpaper]{article}
\usepackage[utf8]{inputenc}
\usepackage{forest}
\usepackage{tikz}
\usetikzlibrary{trees}

%----- Configuración del estilo del documento------%
\usepackage[table]{xcolor}
\usepackage{epsfig,graphicx}
\usepackage[left=2cm,right=2cm,top=1.8cm,bottom=2.3cm]{geometry}
\usepackage{fancyhdr}
\usepackage{lastpage}
\pagestyle{fancy}
\fancyhf{}
\rfoot{\textit{Página \thepage \hspace{1pt} de \pageref{LastPage}}}


%------ Paquetes matemáticos básicos --------%
\usepackage{amsmath}
\usepackage{amssymb}
\usepackage{amsthm}

%------ Texto aleatorio ----- %

\usepackage{lipsum}



\begin{document}

%------ Encabezado -------- %

\begin{center}
    \begin{minipage}{3cm}
    	\begin{center}
    		\includegraphics[height=3.4cm]{./imagenes/logo_unam.png}
    	\end{center}
    \end{minipage}\hfill
    \begin{minipage}{10cm}
    	\begin{center}
    	\textbf{\large Universidad Nacional Autónoma de México}\\[0.1cm]
        \textbf{Facultad de Ciencias}\\[0.1cm]
        \textbf{Estructuras Discretas $|$ Grupo 7020}\\[0.1cm]
        \textbf{Tarea 2: Lógica Preposicional}\\[0.1cm]
        Real Araiza Yamile\\[0.1cm]
        Tenorio Reyes Ihebel Luro\\[0.1cm]
        12/Oct/2024
    	\end{center}
    \end{minipage}\hfill
    \begin{minipage}{3cm}
    	\begin{center}
    		\includegraphics[height=3.4cm]{./imagenes/Logo_FC.png}
    	\end{center}
    \end{minipage}
\end{center}

\rule{17cm}{0.1mm}

%------ Fin de encabezado -------- %

\section*{Indicaciones}

%====== Ejercicio 01 ======%




%====== Ejercicio 02 ======%




%====== Ejercicio 03 ======%




%====== Ejercicio 04 ======%




%====== Ejercicio 05 ======%




%====== Ejercicio 06 ======%
%------   inciso a)  ------%


%------   inciso b)  ------%


%------   inciso c)  ------%


%------   inciso d)  ------%


%------   inciso e)  ------%


%------   inciso f)  ------%


%------   inciso g)  ------%


%------   inciso e)  ------%


%------   inciso h)  ------%


%------   inciso i)  ------%


%------   inciso j)  ------%



%====== Ejercicio 07 ======%
%------   inciso a)  ------%


%------   inciso b)  ------%


%------   inciso c)  ------%


%------   inciso d)  ------%


%------   inciso e)  ------%


%------   inciso f)  ------%


%------   inciso g)  ------%


%------   inciso e)  ------%


%------   inciso h)  ------%


%------   inciso i)  ------%


%------   inciso j)  ------%



\end{document}
 
